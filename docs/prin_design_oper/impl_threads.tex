\subsection{Threads}

To generate parallelism on the node level PThreads are used. This allows
for a very light weight way to to bind a load module or a set of load
modules to a specific CPU set or even a specific CPU. SystemBurn generates
three different types of threads:

\begin{description}

	\item[Scheduler Thread] A single scheduler thread is spawned
	per node. This thread reads the load file and distributes the
	different load modules to the worker threads. The scheduler
	thread also assigns the worker threads to the proper CPU cores
	according to user input from the load file. This thread is also
	responsible for communication with other nodes through MPI calls
	including the communication test load.

	\item[Monitor Thread] A single monitor thread is spawned per
	node. On systems with newer kernels, this thread is in charge
	of collecting temperatures from all of the CPUs on the node
	and updating a thermal state structure which is referenced by
	the scheduler thread. If the monitor thread notes a temperature
	which exceeds the configured maximum, it can "declare" a thermal
	emergency and re-assign the worker threads to a "sleep" load.

	\item[Worker Thread] Worker threads are spawned to execute the
	different load modules specified by the load file. The number
	of worker threads that are able to be spawned is limited by the
	\verb!MAX_WORKER_THREADS! value in the configuration file. If
	the load file specifies more load modules than available worker
	threads, then only the first load modules up to the maximum
	number of worker threads are loaded. The rest of the modules
	are discarded.

\end{description}
