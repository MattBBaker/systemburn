\section{Usage}


Running SystemBurn on a given machine is a four step process.
\begin{itemize}
	\item[Build:] Build the SystemBurn suite. This 
	requires running the \verb!configure! script to set up the suite
	for your system (or manually modifying \verb!Makefile.in! to 
	point to the appropriate libraries and compilers for the target
	system and defining the appropriate macros in \verb!config.h!).
	Once the build is configured simply type:
	\begin{itemize}
		\item \verb!make clean!
		\item \verb!make!
	\end{itemize}
	\item[Config:] Create a configuration file or modify the default
	file, \verb!systemburn.config!. Configuration section of this
	document contains detailed instructions on how to do this.
	\item[Load:] Create a load file or modify the default load in
	\verb!systemburn.load!. This includes determining which modules,
	or plans, to load and how to load them on the architecture.
	The Load Files section of this document contains detailed
	instructions on how to do this.
	\item[Run:] Determine which nodes will be running SystemBurn. If
	a homogeneous node distribution of the load is desired then
	setting up a single MPI tasks per node will be sufficient. If
	a heterogeneous node distribution of the load is desired then
	the following steps must be taken:
	\begin{itemize}
		\item Create all the different loads to be used. This will
		require multiple load files (one for each different load).
		\item Set up a batch script for the system which will
		run different instances of SystemBurn on a set number
		of nodes. (Each instance of SystemBurn will be given
		one of the load files set up in step 1).
		\item Run SystemBurn either from the command line for
		the homogeneous node distribution or with the batch
		script for the heterogeneous node distribution.
	\end{itemize}
\end{itemize}  
